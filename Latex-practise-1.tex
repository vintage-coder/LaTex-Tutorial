\documentclass[12pt]{book}
%\documentclass[12pt]{beamer}
\usepackage{amsmath}
\usepackage{graphicx}
\usepackage{graphics}
\usepackage{subcaption}
\usepackage{mathtools}
\usepackage{xcolor}
%\usepackage[dvipsnames]{xcolor} 
\usepackage{soul} 

\title{My First Document}
\date{2020-09-01}
\author{Gowthaman V}

\begin{document}
%\graphicspath{{C:\Users\linkeddots\Documents\Sample Images\}}


\pagenumbering{gobble}
\maketitle

\newpage

\pagenumbering{Roman}
\noindent
Hello World!. \\ This is my first line.\\
Home page. \\ Welcome to Latex Tutorial.\\
Thanks in Advance.


\section{Preface}
       This is the preface of the entire booklet.
\subsection{Introduction}
       This book has many number of articles.
       
\subsubsection{Basic}
         Basic "hello world" programming in python.\\This is the square root equation or linear equation.
        

       
\begin{equation}   
   f(x)=x^2 
\end{equation}

\begin{equation*}
  F(X)=X^3
\end{equation*}

\begin{align*}
	f(y) &= y^2\\f(y) &= y^3\\f(z) &= z^2 \\f(v)=v^2
\end{align*}

This is the simple linear equation $f(x)=x^2$ example.

\begin{align*}
   f(x) &= x^2\\
   g(x) &= \frac{1}{x}\\
   F(x) &= \int^a_b \frac{1}{3}x^3\\
   G(x) &= \frac{1}{\sqrt{x}}\\
\end{align*}

%\begin{matrix}
%	1 & 0
%  	0 & 1
%\end{matrix}

\begin{align*}
\left(\frac{1}{\sqrt{x}}\right)\\
\lambda
\end{align*}

\newpage

\begin{align*}
\varphi\\
\varpi
\end{align*}


\newpage

\begin{figure}
%\includegraphics[width=\linewidth]{Flag-India.jpg}
%\includegraphics[width=\linewidth,angle=50]{children.jpg}

\includegraphics[width=.2\linewidth]{C:/Users/linkeddots/Documents/Sample Images/hand.jpg}\quad
\includegraphics[width=.2\linewidth]{C:/Users/linkeddots/Documents/Sample Images/hand.jpg}\quad
\includegraphics[width=.2\linewidth]{C:/Users/linkeddots/Documents/Sample Images/hand.jpg}
\\[\baselineskip]
%"C:\Users\linkeddots\Documents\Sample Images\hand.jpg"
\includegraphics[width=.2\linewidth]{C:/Users/linkeddots/Documents/Sample Images/hand.jpg}\quad
\includegraphics[width=.2\linewidth]{C:/Users/linkeddots/Documents/Sample Images/hand.jpg}\quad
\includegraphics[width=.2\linewidth]{C:/Users/linkeddots/Documents/Sample Images/hand.jpg}

\caption{A child}
\label{fig:children}
\end{figure}
Fig \ref{fig:children} shows the children.

\newpage
%...
\begin{figure}[h!]
\centering
\begin{subfigure}[b]{0.4\linewidth}
   \includegraphics[width=\linewidth]{C:/Users/linkeddots/Documents/Sample Images/hand.jpg}
    \caption{children.}
\end{subfigure}

\begin{subfigure}[b]{0.4\linewidth}
   \includegraphics[width=\linewidth]{C:/Users/linkeddots/Documents/Sample Images/hand.jpg}
    \caption{children.}
\end{subfigure}

\caption{Two childrens. Two Times.}
\label{fig:children}

\newpage
\textcolor{green}{\[
\begin{bmatrix}
 a_{11} & b_{12} & c_{13}\\
 d_{21} & e_{22} & f_{23}\\
 g_{31} & h_{32} & i_{33}\\
 \vdots & \ddots & \cdots\\
\end{bmatrix}
\]}

\colorbox{blue}{This is the small matrix example.} $\bigl(\begin{smallmatrix} 1 & 2 & 3\\
4 & 5 & 6
\end{smallmatrix}\bigr)$ to the size of element of \color{pink}the particular matrix.

\end{figure}
%...
This is the sample text.
\textcolor{red} {Learning \LaTeX\ is great experience.} \\ % this is the comment.

\definecolor{gowtham}{gray}{0.75}

\definecolor{orange1}{rgb}{1, 0.5, 1}  
\definecolor{orange2}{RGB}{255, 127, 120}  
\definecolor{orange3}{HTML}{FF7F0F}  
\definecolor{orange4}{cmyk}{0, 1, 0.5,0}

\color{orange1}orange1 \\
\color{orange2}orange2 \\
\color{orange3}orange3 \\
\color{orange4}orange4 \\


\color{gowtham} this is created color.
\pagecolor{yellow}


{\color{red!50!yellow}colored text}\\ % it is a mixture of 50 percent red and 50 percent yellow  
{\color{pink!80!yellow}\LaTeX\ text}\\  
{\color{white!60!red}Some text}\\  
{\color{blue!25!orange}this is text}\\ % it is a mixture of 25 percent blue and 75 percent orange  
{\color{blue!55!orange}Nice text}\\  
{\color{green!70!yellow}Any text}\\  
{\color{blue!20!white!30!green}Beautiful text}\\ % it is a mixture of (20*0.3) percent blue, ((100-20)*0.3) percent white and (100-30) percent green  
{\color{pink!40!red!50!yellow}colored text}  
  
\textcolor{blue}{This the text} \\  


\begin{center}
\textbf{\color{blue}The highlighted text below.}

\end{center}
 This is the \hl{highlighted text}

\section{Chaptor 2}

\subsection{LaTeX Tutorial}

\section{Chaptor 3}

\subsection{Basic Elements of Latex }

\section{Chaptor 4}

\subsection{Intermediate Latex }

\section{Chaptor 5}

\subsection{Advanced Latex }

\end{document}
