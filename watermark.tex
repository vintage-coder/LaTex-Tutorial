\documentclass[12pt]{article}


\usepackage{background}

%\backgroundsetup{contents=Confidential,opacity=0.45,scale=6,color=blue,angle=45}

\backgroundsetup{contents=\includegraphics{science},opacity=0.45,scale=1,color=blue,angle=45}


\title{Watermark in LaTex}
\author{Dong Lee\\ author@gmail.com}
\date{\today}

\begin{document}
\NoBgThispage
\maketitle
\hrule
\clearpage
The original rectangular base of 6 by 10 miles (9.7 by 16.1 km) is now part of the so-called "Groom box", a rectangular area measuring 23 by 25 miles (37 by 40 km), of restricted airspace. The area is connected to the internal Nevada Test Site (NTS) road network, with paved roads leading south to Mercury and west to Yucca Flat. Leading north east from the lake, the wide and well-maintained Groom Lake Road runs through a pass in the Jumbled Hills. The road formerly led to mines in the Groom basin, but has been improved since their closure. Its winding course runs past a security checkpoint, but the restricted area around the base extends farther east. After leaving the restricted area, Groom Lake Road descends eastward to the floor of the Tikaboo Valley, passing the dirt-road entrances to several small ranches, before converging with State Route 375, the "Extraterrestrial Highway", south of Rachel.


\end{document}