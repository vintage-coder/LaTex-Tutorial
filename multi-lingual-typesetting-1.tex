\documentclass[12pt]{article}

\usepackage{polyglossia}
\setdefaultlanguage{english}
\setotherlanguages{hindi,sanskrit,urdu,bengali}

\usepackage{fontspec}
\setmainfont{Times New Roman}

%\newfontfamily{\devanagarifont}[Script=Devanagari]{Lohit-Devanagari.ttf}
%\newfontfamily{\urdufont}[Script=Arabic]{Jameel Noori Nastaleeq.ttf}

\newfontfamily{\devanagarifont}[Script=Devanagari]{NotoSansDevanagari-Regular.ttf}

%\newfontfamily{\bengalifont}[Script=bengali]{kalpurush.ttf}


\title{Hindi,Sanskrit, Urdu, Bengali, and Other Languages in }
\author{Ethan}
\date{03-10-2020}


\begin{document}
\maketitle
\vskip-1cm
\hrulefill

\section{English}
Beautiful and free fonts for all languages
When text is rendered by a computer, sometimes characters are displayed as tofu. They are little boxes to indicate your device doesn have a font to display the text.
Google has been developing a font family called Noto, which aims to support all languages with a harmonious look and feel. Noto is Google answer to tofu. The name noto is to convey the idea that Googlegoal is to see no more tofu. Noto has multiple styles and weights, and is freely available to all. The comprehensive set of fonts and tools used in our development is available in our GitHub repositories.

\section{Hindi and  Sanskrit}
\begin{hindi}
ॐ भूर्भुवः स्वः\\
तत्सवितुर्वरेण्यं\\
भर्गो देवस्य धीमहि\\
धियो यो नः प्रचोदयात् ॥\\
\end{hindi}


\end{document}