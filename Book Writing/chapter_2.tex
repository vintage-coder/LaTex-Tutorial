
\chapter{CHAPTER 2}




\section{LITERATURE SURVEY}

\textbf{ANALYSIS OF THE INTERACTION BETWEEN DIGITAL ART AND TRADITIONAL ART}
The great art form is accomplished from constant accumulations in life, it needs generalization and refining rather than pure imagination. We only have to absorb and integrate the advantages of various art forms in order to create more outstanding works of art that facilitate the audience to accept and recognize. Therefore, digital art and traditional art will obtain better development if they integrate and learn from each other. Especially as the form of new digital art, it should be open-minded to the traditional art of learning, so as to continuously strengthening their creative energy.

\textbf{AN INERTIAL PEN WITH DYNAMIC TIME WARPING RECOGNIZER FOR HANDWRITING AND GESTURE RECOGNITION}

This project presents an inertial-sensor-based digital pen (inertial pen) and its associated dynamic time warping (DTW)-based recognition algorithm for handwriting and gesture recognition. Users hold the inertial pen to write numerals or English lowercase letters and make hand gestures with their preferred handheld style and speed. The proposed DTW-based recognition algorithm includes the procedures of inertial signal acquisition, signal pre-processing, motion detection, template selection, and recognition. We integrate signals collected from an accelerometer, a gyroscope, and a magnetometer into a quaternion-based complementary filter for reducing the integral errors caused by the signal drift or intrinsic noise of the gyroscope, which might reduce the accuracy of the orientation estimation. Furthermore, we have developed a minimal intra-class to maximal inter-class based template selection method (min-max template selection method)for a DTW recognizer to obtain a superior class separation for improved recognition. Experimental results have successfully validated the effectiveness of the DTW-based recognition algorithm for online handwriting and gesture recognition using the inertial pen. 




